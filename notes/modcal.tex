\documentclass{article}

\usepackage{amsmath}
\usepackage{amssymb}
\usepackage{amsthm}
\usepackage{stmaryrd}
%\usepackage{theorem}  %% different theorem syntax
\usepackage{ccicons}
\usepackage{xspace}
\usepackage{ifthen}
\usepackage{array}
\usepackage{enumitem}
\usepackage{bbm}
\usepackage[usenames,dvipsnames]{color}
\usepackage{hyperref}
\hypersetup{colorlinks=true, linkcolor=Red, citecolor=ForestGreen,
  urlcolor=RoyalBlue}
\usepackage{natbib}

\usepackage[all]{xy}
\input{macros}
\renewcommand{\rulename}[1]{\ensuremath{\mathsf{#1}}}

\setlength{\parindent}{0pt}
\setlength{\parskip}{1ex}
\setlist{nosep}
%\setlist{noitemsep}

\theoremstyle{definition}
\newtheorem{example}{Example}
\newtheorem{counterexample}{Counterexample}
\newtheorem{exercise}{Exercise}
\newtheorem{definition}{Definition}
\newtheorem{recipe}{Recipe}
\theoremstyle{plain}
\newtheorem{theorem}{Theorem}
\newtheorem{lemma}{Lemma}
\newtheorem{observation}{Observation}
\newtheorem{fallacy}{Fallacy}
\theoremstyle{remark}
\newtheorem{remark}{Remark}

\newenvironment{proof*}[1][]{\noindent\ifthenelse{\equal{#1}{}}{{\it
      Proof.}}{{\it Proof #1.}}\hspace{2ex}}{\bigskip}


%\newcommand{\Pot}{\mathcal{P}}

\title{A Type-Theoretic Model of Hierarchical Modules}
%\subtitle{Scriptum}
\author{Andreas Abel}
\date{4 May 2019}

\begin{document}
\maketitle

\begin{abstract}
\end{abstract}

\section{Introduction}
\label{sec:intro}

\section{Hierarchical Modules: A Minimal Calculus}

Abstract syntax.
\[
\begin{array}{lrl@{\qquad}l}
  s & ::=  & \mdef x {s^*}    & \mbox{module definition} \\
    & \mid & \mref {x^+}      & \mbox{module reference}  \\[1ex]
  x &  &                      & \mbox{simple name}       \\
  q & ::=  & x^+              & \mbox{qualified name}    \\[1ex]
  \alpha,\beta  & ::= & s^*   & \mbox{module content}    \\
  \Gamma,\Delta & ::= & \Ga^* & \mbox{scope}             \\
\end{array}
\]
A concrete syntax for statements could be the Agda module syntax, \ie,
$\texttt{module}~x~\texttt{where}~s^*$ for $\mdef x {s^*}$, and
$\texttt{module~\_~=}~q$ for $\mref q$.

Sequences of type $\_^*$ may be empty whereas $\_^+$ stands for
non-empty sequences.
We write $\Ge$ for the empty sequence and use a dot %or raised dot
for concatenation.
We silently cast elements to singleton sequences, \eg,
we write $x$ for $x.\Ge$.
We write $\Gamma \chop s$ for the result of splitting the last
statement $s$ off a scope $\Delta \sep (\Ga \sep s) \sep \Ge^*$;
then $\Gamma = \Delta \sep \Ga$.
% This way, $\Gamma \sep s$ makes sense, \eg,
% $(\Delta \sep \alpha) \sep s$ means $\Delta \sep (\alpha \sep s)$,
% and $\Ge \sep s$ (as scope) means $\Ge \sep (\Ge \sep s)$.

Judgements.
\[
\begin{array}{l@{\qquad}l}
  % \Declares s      {x^+} & \mbox{Statement $s$ declares name $x^+$} \\
  % \Declares {s^*}  {x^+} & \mbox{Content $s^*$ declares name $x^+$} \\
  % \Declares \Gamma {x^+} & \mbox{Scope $\Gamma$ declares name $x^+$} \\[1ex]
  \Contains \Gamma q & \mbox{Scope $\Gamma$ declares name $q$} \\
  \Declares \Gamma q & \mbox{Scope $\Gamma$ declares name $q$ (disambiguated)} \\
  \WContent \Gamma \alpha & \mbox{Module content $\alpha$ is well-scoped in $\Gamma$}\\
  \WScope \Gamma & \mbox{Scope $\Gamma$ is well-formed}
\end{array}
\]
\fbox{$\Contains \Gamma q$}
\begin{gather*}
  \nru{\rhere
     }{
     }{\Contains {\Gamma \chop \mdef x \beta} x}
\qquad
  \nru{\rinto
     }{\Contains \beta q
     }{\Contains {\Gamma \chop \mdef x \beta} {x.q}}
\qquad
   \nru{\rthere
      }{\Contains \Gamma q
      }{\Contains {\Gamma \chop s} q}
\end{gather*}
The judgement $\Contains \Gamma q$ is a decidable set, its inhabitants
are paths to the definition site of a name.
% A name $q$ is
% \emph{ambiguous} in scope $\Gamma$ if $\Contains \Gamma q$ has more
% than one inhabitant.
$\Contains \Gamma q$ may have more than one inhabitant, meaning that
name $q$ would be ambiguous.
We can disambiguate by always take the closest declaration as the
definition of a name.

\fbox{$\Declares \Gamma q$} (implies $\Contains \Gamma q$).
\begin{gather*}
  \nru{\rhere
     }{
     }{\Declares {\Gamma \chop \mdef x \beta} x}
\qquad
  \nru{\rinto
     }{\Declares \beta q
     }{\Declares {\Gamma \chop \mdef x \beta} {x.q}}
\qquad
   \nru{\rthere
      }{\Declares \Gamma q \qquad \NotContains s q
      }{\Declares {\Gamma \chop s} q}
\end{gather*}
$\Declares \Gamma q$ is a proposition, \ie, contains at most one
inhabitant.  This inhabitant, when it exists, may be the considered
the \emph{resolution} of name $q$ in $\Gamma$, and it is the
equivalent of a de Bruijn index in lambda calculus.
\begin{lemma}[Unambiguity of resolved names]
  If $y,z : (\Declares \Gamma q)$ then $y = z$.
\end{lemma}
\begin{proof}
  By induction on $y,z : (\Declares \Gamma q)$.
  For a representative case, consider
  $y = \rinto\;y'$ and $z = \rthere\;z'$.
  Assuming the last declaration in $\Gamma$ is $s = \mdef x \beta$,
  this means that $y' : (\Declares \beta q)$,
  but also $\NotContains s q$, implied by $z$.  However,
  $\rinto\;y' : (\Declares s q)$, which is impossible, as it implies
  $\Contains s q$.
\end{proof}

Well-scopedness of statement (lists): propositions
\fbox{$\WContent \Gamma s$}  and
\fbox{$\WContent \Gamma \alpha$}\,.
\begin{gather*}
  \ru{\Declares \Gamma q}{\WContent \Gamma {\mref q}}
\qquad
  \ru{\NoClash \Gamma x \qquad \WContent \Gamma \beta
    }{\WContent \Gamma {\mdef x \beta}
    }
\qquad
  \ru{}{\WContent \Gamma \Ge}
\qquad
  \ru{\WContent \Gamma \alpha \qquad
      \WContent {\Gamma \sep \alpha} s
    }{\WContent \Gamma {\alpha \sep s}
    }
\end{gather*}
Declaring a new name $x$ involves checking for name clashes via judgement
$\NoClash \Gamma x$.  We can have $\NoClash \Gamma x$ vacuously true
without introducing ambiguity, but this might permit unwanted
shadowing.
For instance, content $\mdef x \alpha \sep \mdef x \beta \sep \mref x$ may be ruled out
since it introduces the same name $x$ \emph{on the same level} twice.
The reference $x$ is still unambiguous as we resolve it to the last
definition of $x$, however, this is likely to confuse programmers
working in a language with such shadowing.
In contrast, $\mdefp x {\mdef x \Ge}$ which declares a
parent $x$ with a child named $x$, is less controversial, since the
shadowing is on different levels.  It is very common that local
definitions may share global definitions, for instance.

To disallow shadowing on the same level, we define proposition
\fbox{$\NoClash \Gamma x$} as follows, using auxiliary proposition
\fbox{$\NoClash \alpha x$} and
\fbox{$\NoClash s x$}.
\begin{gather*}
  \ru{}{\NoClash \Ge x}
\qquad
  \ru{\NoClash \alpha x
    }{\NoClash {\Gamma \sep \alpha} x}
\qquad
  \ru{\NoClash \alpha x \qquad \NoClash s x
    }{\NoClash {\alpha \sep s} x}
\qquad
%\\[2ex]
  \ru{}{\NoClash {\mref q} x}
\qquad
  \ru{x \not= x'
    }{\NoClash {\mdef {x'} \beta} x}
\end{gather*}
Note that for $\NoClash {\Gamma \sep \alpha} x$,
we only check the last content block $\alpha$,
which is on the same level as the to-be-defined name $x$.
However, then we need to check \emph{all} statements within that block
$\alpha$.

Well-formedness of scopes $\WScope \Gamma$.
\begin{gather*}
  \ru{}{\WScope \Ge}
\qquad
  \ru{\WScope \Gamma \qquad \WContent \Gamma \alpha
    }{\WScope {\Gamma \sep \alpha}}
\end{gather*}

\section{Discussion}

\section{Related Work}

\paragraph*{Acknowledgments}

\bibliographystyle{plainurl}
\bibliography{all}

\end{document}
