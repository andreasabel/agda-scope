\documentclass{article}

\usepackage{amsmath}
\usepackage{amssymb}
\usepackage{amsthm}
\usepackage{stmaryrd}
%\usepackage{theorem}  %% different theorem syntax
\usepackage{ccicons}
\usepackage{xspace}
\usepackage{ifthen}
\usepackage{array}
\usepackage{enumitem}
\usepackage{bbm}
\usepackage[usenames,dvipsnames]{color}
\usepackage{hyperref}
\hypersetup{colorlinks=true, linkcolor=Red, citecolor=ForestGreen,
  urlcolor=RoyalBlue}
\usepackage{natbib}

\usepackage[all]{xy}
\input{macros}
\renewcommand{\rulename}[1]{\ensuremath{\mathsf{#1}}}

\setlength{\parindent}{0pt}
\setlength{\parskip}{1ex}
\setlist{nosep}
%\setlist{noitemsep}

\theoremstyle{definition}
\newtheorem{example}{Example}
\newtheorem{counterexample}{Counterexample}
\newtheorem{exercise}{Exercise}
\newtheorem{definition}{Definition}
\newtheorem{recipe}{Recipe}
\theoremstyle{plain}
\newtheorem{theorem}{Theorem}
\newtheorem{lemma}{Lemma}
\newtheorem{observation}{Observation}
\newtheorem{fallacy}{Fallacy}
\theoremstyle{remark}
\newtheorem{remark}{Remark}

\newenvironment{proof*}[1][]{\noindent\ifthenelse{\equal{#1}{}}{{\it
      Proof.}}{{\it Proof #1.}}\hspace{2ex}}{\bigskip}


%\newcommand{\Pot}{\mathcal{P}}

\title{A Type-Theoretic Model of Hierarchical Modules}
%\subtitle{Scriptum}
\author{Andreas Abel}
\date{4 May 2019}

\begin{document}
\maketitle

\begin{abstract}
\end{abstract}

\section{Introduction}
\label{sec:intro}

\section{Hierarchical Modules: A Minimal Calculus}

Disclaimer: may be outdated, the next section is actively worked on.

Abstract syntax.
\[
\begin{array}{lrl@{\qquad}l}
  s & ::=  & \mdef x {s^*}    & \mbox{module definition} \\
    & \mid & \mref {x^+}      & \mbox{module reference}  \\[1ex]
  x &  &                      & \mbox{simple name}       \\
  q & ::=  & x^+              & \mbox{qualified name}    \\[1ex]
  \alpha,\beta  & ::= & s^*   & \mbox{module content}    \\
  \Gamma,\Delta & ::= & \Ga^* & \mbox{scope}             \\
\end{array}
\]
A concrete syntax for statements could be the Agda module syntax, \ie,
$\texttt{module}~x~\texttt{where}~s^*$ for $\mdef x {s^*}$, and
$\texttt{module~\_~=}~q$ for $\mref q$.

Sequences of type $\_^*$ may be empty whereas $\_^+$ stands for
non-empty sequences.
We write $\Ge$ for the empty sequence and use a dot %or raised dot
for concatenation.
We silently cast elements to singleton sequences, \eg,
we write $x$ for $x.\Ge$.
We write $\Gamma \chop s$ for the result of splitting the last
statement $s$ off a scope $\Delta \sep (\Ga \sep s) \sep \Ge^*$;
then $\Gamma = \Delta \sep \Ga$.
% This way, $\Gamma \sep s$ makes sense, \eg,
% $(\Delta \sep \alpha) \sep s$ means $\Delta \sep (\alpha \sep s)$,
% and $\Ge \sep s$ (as scope) means $\Ge \sep (\Ge \sep s)$.

Judgements.
\[
\begin{array}{l@{\qquad}l}
  % \Declares s      {x^+} & \mbox{Statement $s$ declares name $x^+$} \\
  % \Declares {s^*}  {x^+} & \mbox{Content $s^*$ declares name $x^+$} \\
  % \Declares \Gamma {x^+} & \mbox{Scope $\Gamma$ declares name $x^+$} \\[1ex]
  \Contains \Gamma q & \mbox{Scope $\Gamma$ declares name $q$} \\
  \Declares \Gamma q & \mbox{Scope $\Gamma$ declares name $q$ (disambiguated)} \\
  \WContent \Gamma \alpha & \mbox{Module content $\alpha$ is well-scoped in $\Gamma$}\\
  \WScope \Gamma & \mbox{Scope $\Gamma$ is well-formed}
\end{array}
\]
\fbox{$\Contains \Gamma q$}
\begin{gather*}
  \nru{\rhere
     }{
     }{\Contains {\Gamma \chop \mdef x \beta} x}
\qquad
  \nru{\rinto
     }{\Contains \beta q
     }{\Contains {\Gamma \chop \mdef x \beta} {x.q}}
\qquad
   \nru{\rthere
      }{\Contains \Gamma q
      }{\Contains {\Gamma \chop s} q}
\end{gather*}
The judgement $\Contains \Gamma q$ is a decidable set, its inhabitants
are paths to the definition site of a name.
% A name $q$ is
% \emph{ambiguous} in scope $\Gamma$ if $\Contains \Gamma q$ has more
% than one inhabitant.
$\Contains \Gamma q$ may have more than one inhabitant, meaning that
name $q$ would be ambiguous.
We can disambiguate by always take the closest declaration as the
definition of a name.

\fbox{$\Declares \Gamma q$} (implies $\Contains \Gamma q$).
\begin{gather*}
  \nru{\rhere
     }{
     }{\Declares {\Gamma \chop \mdef x \beta} x}
\qquad
  \nru{\rinto
     }{\Declares \beta q
     }{\Declares {\Gamma \chop \mdef x \beta} {x.q}}
\qquad
   \nru{\rthere
      }{\Declares \Gamma q \qquad \NotContains s q
      }{\Declares {\Gamma \chop s} q}
\end{gather*}
$\Declares \Gamma q$ is a proposition, \ie, contains at most one
inhabitant.  This inhabitant, when it exists, may be the considered
the \emph{resolution} of name $q$ in $\Gamma$, and it is the
equivalent of a de Bruijn index in lambda calculus.
\begin{lemma}[Unambiguity of resolved names]
  If $y,z : (\Declares \Gamma q)$ then $y = z$.
\end{lemma}
\begin{proof}
  By induction on $y,z : (\Declares \Gamma q)$.
  For a representative case, consider
  $y = \rinto\;y'$ and $z = \rthere\;z'$.
  Assuming the last declaration in $\Gamma$ is $s = \mdef x \beta$,
  this means that $y' : (\Declares \beta q)$,
  but also $\NotContains s q$, implied by $z$.  However,
  $\rinto\;y' : (\Declares s q)$, which is impossible, as it implies
  $\Contains s q$.
\end{proof}

Well-scopedness of statement (lists): propositions
\fbox{$\WContent \Gamma s$}  and
\fbox{$\WContent \Gamma \alpha$}\,.
\begin{gather*}
  \ru{\Declares \Gamma q}{\WContent \Gamma {\mref q}}
\qquad
  \ru{\NoClash \Gamma x \qquad \WContent \Gamma \beta
    }{\WContent \Gamma {\mdef x \beta}
    }
\qquad
  \ru{}{\WContent \Gamma \Ge}
\qquad
  \ru{\WContent \Gamma \alpha \qquad
      \WContent {\Gamma \sep \alpha} s
    }{\WContent \Gamma {\alpha \sep s}
    }
\end{gather*}
Declaring a new name $x$ involves checking for name clashes via judgement
$\NoClash \Gamma x$.  We can have $\NoClash \Gamma x$ vacuously true
without introducing ambiguity, but this might permit unwanted
shadowing.
For instance, content $\mdef x \alpha \sep \mdef x \beta \sep \mref x$ may be ruled out
since it introduces the same name $x$ \emph{on the same level} twice.
The reference $x$ is still unambiguous as we resolve it to the last
definition of $x$, however, this is likely to confuse programmers
working in a language with such shadowing.
In contrast, $\mdefp x {\mdef x \Ge}$ which declares a
parent $x$ with a child named $x$, is less controversial, since the
shadowing is on different levels.  It is very common that local
definitions may share global definitions, for instance.

To disallow shadowing on the same level, we define proposition
\fbox{$\NoClash \Gamma x$} as follows, using auxiliary proposition
\fbox{$\NoClash \alpha x$} and
\fbox{$\NoClash s x$}.
\begin{gather*}
  \ru{}{\NoClash \Ge x}
\qquad
  \ru{\NoClash \alpha x
    }{\NoClash {\Gamma \sep \alpha} x}
\qquad
  \ru{\NoClash \alpha x \qquad \NoClash s x
    }{\NoClash {\alpha \sep s} x}
\qquad
%\\[2ex]
  \ru{}{\NoClash {\mref q} x}
\qquad
  \ru{x \not= x'
    }{\NoClash {\mdef {x'} \beta} x}
\end{gather*}
Note that for $\NoClash {\Gamma \sep \alpha} x$,
we only check the last content block $\alpha$,
which is on the same level as the to-be-defined name $x$.
However, then we need to check \emph{all} statements within that block
$\alpha$.

Well-formedness of scopes $\WScope \Gamma$.
\begin{gather*}
  \ru{}{\WScope \Ge}
\qquad
  \ru{\WScope \Gamma \qquad \WContent \Gamma \alpha
    }{\WScope {\Gamma \sep \alpha}}
\end{gather*}

%%%%%%%%%%%%%%%%%%%%%%%%%%%%%%%%%%%%%%%%%%%%%%%%%%%%%%%%%%%%%%%%%%%%%%%%%%%


\section{Hierarchical Modules: A Minimal Calculus With Aliases}

Disclaimer: cut-and-paste from previous section, repetitive.

Abstract syntax.
\[
\begin{array}{lrl@{\qquad}l}
  s & ::=  & \mdef x {s^*}    & \mbox{module definition} \\
    & \mid & \malias x q      & \mbox{module alias}  \\[1ex]
  x &  &                      & \mbox{simple name}       \\
  q & ::=  & x^+              & \mbox{qualified name}    \\[1ex]
  \alpha,\beta  & ::= & s^*   & \mbox{module content}    \\
  \Gamma,\Delta & ::= & \Ga^* & \mbox{scope}             \\
\end{array}
\]
A concrete syntax for statements could be the Agda module syntax, \ie,
$\texttt{module}~x~\texttt{where}~s^*$ for $\mdef x {s^*}$, and
$\texttt{module}~x~\texttt{=}~q$ for $\malias x q$.

Another interpretation would be a file system with folders only.  Then
$\mdef x {s^*}$ defines a folder with subfolders $s^*$, and $\malias x
q$ is a symbolic link to $q$ named $x$.  However, file systems usually
allow arbitrary recursion, \eg, we can have folders $A$ and $B$ where
$A$ contains a link to $B$ and vice versa.  This would yield an
infinite path $A.B.A.B...$. In contrast, we are aiming
at well-founded structures, where pathes are finite and
aliases can in principle be expanded.

Sequences of type $\_^*$ may be empty whereas $\_^+$ stands for
non-empty sequences.
We write $\Ge$ for the empty sequence and use a dot %or raised dot
for concatenation.
We silently cast elements to singleton sequences, \eg,
we write $x$ for $x.\Ge$.
We write \fbox{$\Gamma \chop s$} for the result of splitting the last
statement $s$ off a scope $\Delta \sep (\Ga \sep s) \sep \Ge^*$;
then $\Gamma = \Delta \sep \Ga$.
% This way, $\Gamma \sep s$ makes sense, \eg,
% $(\Delta \sep \alpha) \sep s$ means $\Delta \sep (\alpha \sep s)$,
% and $\Ge \sep s$ (as scope) means $\Ge \sep (\Ge \sep s)$.
The name $x$ defined in statement $s$ is \fbox{$\defined s$}, thus,
$\defined(\mdef x \beta) = \defined(\malias x q) = x$.

Judgements.
\[
\begin{array}{l@{\qquad}l}
  % \Declares s      {x^+} & \mbox{Statement $s$ declares name $x^+$} \\
  % \Declares {s^*}  {x^+} & \mbox{Content $s^*$ declares name $x^+$} \\
  % \Declares \Gamma {x^+} & \mbox{Scope $\Gamma$ declares name $x^+$} \\[1ex]
  \Contains \Gamma q & \mbox{Scope $\Gamma$ declares name $q$} \\
  \Declares \Gamma q & \mbox{Scope $\Gamma$ declares name $q$ (disambiguated)} \\
  \WContent \Gamma \alpha & \mbox{Module content $\alpha$ is well-scoped in $\Gamma$}\\
  \WScope \Gamma & \mbox{Scope $\Gamma$ is well-formed}
\end{array}
\]
\fbox{$\Contains \Gamma q$}
\begin{gather*}
  \nru{\rhere
     }{
     }{\Contains {\Gamma \chop s} {\defined s}}
\qquad
   \nru{\rthere
      }{\Contains \Gamma q
      }{\Contains {\Gamma \chop s} q}
\\[2ex]
  \nru{\rinto
     }{\Contains \beta q
     }{\Contains {\Gamma \chop \mdef x \beta} {x.q}}
\qquad
  \nru{\rfollow
     }{\Contains \Gamma {q'.q}
     }{\Contains {\Gamma \chop \malias x q'} {x.q}}
\end{gather*}
The judgement $\Contains \Gamma q$ is a decidable set, its inhabitants
are paths to the definition site of a name.

To resolve a qualified name $x.q$ matching an alias $q$,
$\Contains {\Gamma \chop \malias x q'} {x.q}$,
the intuitive procedure would be to look up the content $\alpha$
of the aliased module $q'$, and then resolve $q'$ in $\alpha$.
However, the decidability would be less trivial; for termination,
we would have to argue that $\alpha$ is structurally smaller
than $\Gamma$.  While this is easy to see, it still would be more
laborious to formalize: we would need a relation that maps a name to
its content (if any), define a \emph{smaller-than} relation on scopes,
prove its well-foundedness, and prove that in scope $\Gamma$,
the retrieved content $\alpha$ is always smaller than $\Gamma$.
Our $\rfollow$ rule instead expands the alias on the fly, and
continues with the expanded name.


% A name $q$ is
% \emph{ambiguous} in scope $\Gamma$ if $\Contains \Gamma q$ has more
% than one inhabitant.
For fixed $\Gamma$ and $q$, the set $\Contains \Gamma q$
may have more than one inhabitant, meaning that
name $q$ would be ambiguous.
We can disambiguate by always take the closest declaration as the
definition of a name.

\fbox{$\Declares \Gamma q$} (implies $\Contains \Gamma q$).
\begin{gather*}
  \nru{\rhere
     }{
     }{\Declares {\Gamma \chop s} {\defined s}}
\qquad
   \nru{\rthere
      }{\Declares \Gamma q \qquad \NotContains s q
      }{\Declares {\Gamma \chop s} q}
\\[2ex]
  \nru{\rinto
     }{\Declares \beta q
     }{\Declares {\Gamma \chop \mdef x \beta} {x.q}}
\qquad
  \nru{\rfollow
     }{\Declares \Gamma {q'.q}
     }{\Declares {\Gamma \chop \malias x q'} {x.q}}
\end{gather*}
$\Declares \Gamma q$ is a proposition, \ie, contains at most one
inhabitant.  This inhabitant, when it exists, may be the considered
the \emph{resolution} of name $q$ in $\Gamma$, and it is the
equivalent of a de Bruijn index in lambda calculus.
\begin{lemma}[Unambiguity of resolved names]
  If $y,z : (\Declares \Gamma q)$ then $y = z$.
\end{lemma}
\begin{proof}
  By induction on $y,z : (\Declares \Gamma q)$.
  For a representative case, consider
  $y = \rinto\;y'$ and $z = \rthere\;z'$.
  Assuming the last declaration in $\Gamma$ is $s = \mdef x \beta$,
  this means that $y' : (\Declares \beta q)$,
  but also $\NotContains s q$, implied by $z$.  However,
  $\rinto\;y' : (\Declares s q)$, which is impossible, as it implies
  $\Contains s q$.
\end{proof}

Well-scopedness of statement (lists): propositions
\fbox{$\WContent \Gamma s$}  and
\fbox{$\WContent \Gamma \alpha$}\,.
\begin{gather*}
  \ru{\NoClash \Gamma x \qquad \Declares \Gamma q
    }{\WContent \Gamma {\malias x q}}
\qquad
  \ru{\NoClash \Gamma x \qquad \WContent \Gamma \beta
    }{\WContent \Gamma {\mdef x \beta}
    }
\qquad
  \ru{}{\WContent \Gamma \Ge}
\qquad
  \ru{\WContent \Gamma \alpha \qquad
      \WContent {\Gamma \sep \alpha} s
    }{\WContent \Gamma {\alpha \sep s}
    }
\end{gather*}
Declaring a new name $x$ involves checking for name clashes via judgement
$\NoClash \Gamma x$.  We can have $\NoClash \Gamma x$ vacuously true
without introducing ambiguity, but this might permit unwanted
shadowing.
For instance, content $\mdef x \alpha \sep \mdef x \beta \sep \malias y x$ may be ruled out
since it introduces the same name $x$ \emph{on the same level} twice.
The reference $x$ is still unambiguous as we resolve it to the last
definition of $x$, however, this is likely to confuse programmers
working in a language with such shadowing.
In contrast, $\mdefp x {\mdef x \Ge}$ which declares a
parent $x$ with a child named $x$, is less controversial, since the
shadowing is on different levels.  It is very common that local
definitions may share global definitions, for instance.

To disallow shadowing on the same level, we define proposition
\fbox{$\NoClash \Gamma x$} as follows:
\begin{gather*}
  \ru{}{\NoClash \Ge x}
\qquad
  \ru{\NotDeclares \alpha x
    }{\NoClash {\Gamma \sep \alpha} x}
\end{gather*}
Note that for $\NoClash {\Gamma \sep \alpha} x$,
we only check the last content block $\alpha$,
which is on the same level as the to-be-defined name $x$.
(However, we then check \emph{all} statements within that block
$\alpha$.)

Well-formedness of scopes $\WScope \Gamma$.
\begin{gather*}
  \ru{}{\WScope \Ge}
\qquad
  \ru{\WScope \Gamma \qquad \WContent \Gamma \alpha
    }{\WScope {\Gamma \sep \alpha}}
\end{gather*}

\section{Opening Modules}

Module systems often feature statements that let us include the
contents of one module in another.  In Agda, this is the \texttt{open}
statement.
\[
\begin{array}{lrl@{\qquad}l}
  s & ::= & \mopen q & \mbox{import content of module $q$}
\end{array}
\]
In the most basic form, where $\mopen q$ imports the \emph{whole}
content of module $q$ into the current scope, $\topen$ can be realized
analogously to $\talias$:
\[
  \nru{\ropen
     }{\Contains \Gamma {q'.q}
     }{\Contains {\Gamma \sep \mopen {q'}} q}
\qquad
  \nru{\ropen
     }{\Declares \Gamma {q'.q}
     }{\Declares {\Gamma \sep \mopen {q'}} q}
\qquad
  \ru{\Declares \Gamma q
    }{\WContent \Gamma {\mopen q}}
\]
In fact, $\topen$ is just an $\talias$ with an empty name.
We might consider aliases that define \emph{qualified} names,
including the empty name.
\[
\begin{array}{lrl@{\qquad}l}
  s & ::= & \malias \vxs q & \mbox{import content of module $q$ under
                              extra qualification $\vxs$} \\
  \vxs & ::= & x^*         & \mbox{qualification prefix} \\
\end{array}
\]
Then $\mopen q = \malias \Ge q$.

Setting $\defined(\malias \vxs q) = \vxs$ updates the $\rhere$
rule.  The $\rfollow$ rule is virtually unchanged, just generalized
from $x$ to $\vxs$.
\[
  \nru{\rfollow
     }{\Contains \Gamma {q'.q}
     }{\Contains {\Gamma \chop \malias \vxs q'} {\vxs.q}}
\qquad
  \nru{\rfollow
     }{\Declares \Gamma {q'.q}
     }{\Declares {\Gamma \chop \malias \vxs q'} {\vxs.q}}
\]
It is a bit unclear though what the shadowing rules for aliases $\vxs$
should be.

The simulation of $\topen$ via a generalized $\talias$ can be
reversed: we can encode $\malias x q$ as $\mdefp x {\mopen q}$.
Similarily, a generalized $\malias {x_1\dots x_n} q$ where $n \geq 0$
can be encoded by $\mdefp {x_1} {\dots \mdefp {x_n} {\mopen q}}$.
It seems thus that $\topen$ is more primitive than $\talias$, and we
can drop $\talias$ from the core module calculus.


\subsection{Ambiguity}

The names brought into scope by $\topen$ might clash with names
already in scope.  Thus with $\topen$, the judgement
$\Declares \Gamma q$ is no longer a proposition in general.
However, whenever we \emph{use} a name $q$,
we require it to be non-ambiguous.


\subsection{Import restriction}

It is very common that imports can be specified precisely, \ie, we may
only import certain identifiers from a module.  Agda and Haskell have
import directives for selecting only certain identifiers (Agda keyword
\texttt{using}) or selecting all but certain identifiers (Agda and
Haskell keyword \texttt{hiding}).  Agda also allows \texttt{renaming}
of identifiers.  These directives can be subsumed under a
\emph{renaming map} $\rho$ that maps simple names to simple names.
The imported symbols are the domain of $\rho$, their names are
given by the range of $\rho$.  The map should be injective, \ie, two
different names should not be mapped to the same name.  It need not
necessarily be functional, \ie, nothing speaks against importing a name
twice under different names.\footnote{Agda forbids duplicate imports
  of the same symbol (even when under different name) in a single
  \texttt{open} statement.
  As a consequence $\rho$, can be stored as a finite map from
  imported symbols to their new name.
  One may want to argue in favor of Agda
  that importing the same symbol twice
  (while under different names) is non-sensical and could be
  errorneous and unintended by the user.  Agda does allow repetitive
  import of the same symbol in separate \texttt{open} statements,
  though.  One may want to argue further that this way, the user states more
  explicitely that they want to import the same symbol twice.
  However, one may argue just the other way round: a second
  \texttt{open} statement with the same explicit import
  is given in error, and stating the duplicate import
  twice in the same \texttt{open} statement emphasizes the intent of
  the user.}
It might thus be more elegant to
consider $\rho$ as a map from names defined by the import to their
origin.  Thus, its range consists of the imported symbols, and its
domain gives their new names.
\[
\begin{array}{lrl@{\qquad}l}
  s & ::= & \mopenr q \rho & \mbox{import content, as given by $\rho$, of module $q$}
\end{array}
\]
When trying to resolve a qualified name $x.\vxs$ in $\mopenr {q} \rho$, we
subject $x$ to the renaming $\rho$.
\begin{gather*}
  \nru{\ropen
     }{\Contains \Gamma {q.\rho(x).\vxs}
     }{\Contains {\Gamma \sep \mopenr {q} \rho} {x.\vxs}}
\qquad
  \nru{\ropen
     }{\Declares \Gamma {q.\rho(x).\vxs}
     }{\Declares {\Gamma \sep \mopenr {q} \rho} {x.\vxs}}
\\[2ex]
  \ru{\Declares \Gamma q  \qquad
      \forall x:\rng \rho.\ \Declares \Gamma {q.x}
    }{\WContent \Gamma {\mopenr q \rho}}
\end{gather*}
An $\topen$ statement is well-formed if it does not try to import
non-existing symbols.  In Agda-2.6.0, a violation of this sanity
condition is not a hard error, it only produces a warning.

\section{Discussion}

\section{Related Work}

\paragraph*{Acknowledgments}

\bibliographystyle{plainurl}
\bibliography{all}

\end{document}
